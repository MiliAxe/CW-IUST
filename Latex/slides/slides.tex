\documentclass{beamer}

\usepackage{showexpl}

\lstset
{
    language=[LaTeX]TeX,
    breaklines=true,
    basicstyle=\tt\scriptsize,
    keywordstyle=\color{blue},
    identifierstyle=\color{magenta},
    numberstyle=\tiny\color{gray},
    tabsize=2,
    numbers=left,                    
    numbersep=5pt,
}

\usetheme{Singapore}

\title{Computer Workshop}
\subtitle{\LaTeX}
\author{Dr. MalekiMajd}
\institute[IUST]{Iran University Of Science And Technology}
\date{24 Aban, 1402}

\begin{document}

\frame{\titlepage}

\begin{frame}
\frametitle{End Goals}
\begin{enumerate}
    \item Get to know about the LaTeX workflow and its advantages.
    \item Be able to produce high-quality and professional documents.
    \item Be able to produce academic-ready documents without hassle.
\end{enumerate}
\end{frame}

\begin{frame}{What is \LaTeX?}
\LaTeX, which is pronounced "Lah-tech" or "Lay-tech", is a document preparation system for high-quality typesetting. It is most often used for medium-to-large technical or scientific documents, but it can be used for almost any form of publishing.


LaTeX is not a word processor! Instead, LaTeX encourages authors not to worry too much about the appearance of their documents but to concentrate on getting the right content.
\end{frame}

\begin{frame}{Why use \LaTeX?}
\begin{itemize}
    \item Allows you to focus on the \textit{content} of the document rather than worrying about the formatting.
    \item Provides the creation of high quality documents.
    \item Extremely customizable.
    \item Your document is extremely stable. No matter how complex your documents are.
    \item Extremely extensible with the use of packages.
\end{itemize}
\end{frame}

\begin{frame}{Getting started}
In order to convert your \LaTeX \ files into a human-readable format, you have to compile it.
You generally have the ways down below to compile your \LaTeX \ files:
\begin{enumerate}
    \item Install a distribution and use the terminal to compile your documents. (TexLive is recommended)
    \item Use an Online editor. (Overleaf)
    \item Use a preconfigured editor (e.g. TeXstudio)
\end{enumerate}
\end{frame}

\begin{frame}[fragile]{Creating our first document}
Let's start by creating our very basic document.

\begin{LTXexample}
\documentclass{article}
\begin{document}
Hello from the document!
\end{document}
\end{LTXexample}

\begin{itemize}
    \item \lstinline!\documentclass! specifies the document type, it can have different types like
        \texttt{article, book, report...}
    \item \lstinline!\begin{enviroment}\end{enviroment}! Starts and ends an environment respectively. 
        Enviroments apply a certain formatting to the content inside them. Here, we start our document
        with the \texttt{document} environment. This is where our document content goes.
\end{itemize}

\end{frame}

\begin{frame}{Commands in \LaTeX}
    Commands give you the power to further customize your document. Most LaTeX commands are simple words
    preceded by a backslash character (\textbackslash).
    
    You can also put comments in your LaTeX file by using a \texttt{\%}.
\end{frame}

\begin{frame}{The preamble}
In the previous slide we found out that the document content goes into the \texttt{document}
environment. We also specified the document type before starting our document content. This
part of the document is called the \textit{preamble}. It is essentially a place to further customize
your document.
\end{frame}

\begin{frame}[fragile]{Adding a title page}

Now, let's see how we can add a title page to our document. This is how it's done:
\begin{enumerate}
    \item Define the title, author, date and further information in the preamble.
    \item Create the title using \lstinline!\maketitle! command.
    \item Tell the compiler to allocate a whole page for the title page.
\end{enumerate}
\end{frame}

\begin{frame}[fragile]{Adding a title page}
\begin{LTXexample}
\documentclass{article}

\title{Sample LaTeX document}
\author{CW Class}
\institute{IUST}
\date{24 Aban}

\begin{document}
\maketitle
Hello from the document!
\end{document}
\end{LTXexample}
\end{frame}

\begin{frame}[fragile]{One-page title}
You might notice that here, our title does not have a whole page allocated to it. In this case, LaTeX
gives us the option to to add extra options to our \lstinline!\documentclass!:

\begin{lstlisting}
\documentclass[titlepage]{article}
\end{lstlisting}
You are slowly getting the hang of the syntax in LaTeX here, you might encounter this syntax in other
places where you pass extra options to a command using the brackets.
\end{frame}

\begin{frame}[fragile]{Sections}
You can structure your document by giving it sections, subsections, and more.
\begin{lstlisting}
\section{A section}
\subsection{A subsection}
\subsubsection{A subsubsection}
\paragraph{A paragraph}
\subparagraph{A subparagraph}
\end{lstlisting}
LaTeX by default will take care of the numbering so, you have more focus on the content.

You can make the sections unnumbered by adding an asterisk after the command, before starting the name of
the section: \lstinline!\section*{An unnumbered section}!
\end{frame}


\begin{frame}[fragile]{Starting a New Paragraph}
To start a new paragraph in \LaTeX, leave a blank line in the source code or use the 
\texttt{\textbackslash par} command. This creates a visual separation between paragraphs in the output document.

You might notice that no matter how many spaces you put between your lines in the source code, no
new lines will be added.

\begin{lstlisting}
\section{Test section}
This is the content inside the test section.


Extra new lines will have no effect!
\end{lstlisting}

\end{frame}

\begin{frame}[fragile]{Adding a new line}
    In case you want to add multiple new lines after each other, you can use the \texttt{\textbackslash newline} command.
\begin{example}
\begin{LTXexample}
These \newline are \newline \newline some lines!
\end{LTXexample}
\end{example}

\end{frame}

\begin{frame}
  \frametitle{Text Formatting in \LaTeX: Bold, Italic, Underline}
  Formatting your text to different styles should be pretty straight forward in \LaTeX, you
  just need to use the following commands:
\end{frame}

\begin{frame}{Bold Text}
  To make text \textbf{bold} in \LaTeX, use the \texttt{\textbackslash textbf\{\}} command.

  \begin{example}
    \texttt{\textbackslash textbf\{This text is bold.\}}
  \end{example}
\end{frame}

\begin{frame}{Italic Text}
  To make text \textit{italic} in \LaTeX, use the \texttt{\textbackslash textit\{\}} command.

  \begin{example}
    \texttt{\textbackslash textit\{This text is italic.\}}
  \end{example}
\end{frame}

\begin{frame}{Underline Text}
    To \underline{underline} text in \LaTeX, you can use the \texttt{\textbackslash underline\{\}} command.

  \begin{example}
    \texttt{\textbackslash underline\{This text is underlined.\}}
  \end{example}

\end{frame}


\begin{frame}
  \frametitle{Enumerate and Itemize in \LaTeX}
  This is where environments come handy. You specify a specific environment for your content to be in.
  In this case, we want our content to be in a bulleted list or a numbered list. Let's take a look how
  we can achieve that.
\end{frame}

\begin{frame}[fragile]{Itemize Environment}
  The \texttt{itemize} environment is used for creating bulleted lists. Each item is specified with the \texttt{\textbackslash item} command.

  \begin{example}
    \begin{LTXexample}
    \begin{itemize}
      \item First item
      \item Second item
      \item Third item
    \end{itemize}
    \end{LTXexample}
  \end{example}
\end{frame}

\begin{frame}[fragile]{Enumerate Environment}
  The \texttt{enumerate} environment is used for creating numbered lists. Each item is specified with the \texttt{\textbackslash item} command.

  \begin{example}
    \begin{LTXexample}
    \begin{enumerate}
      \item First item
      \item Second item
      \item Third item
    \end{enumerate}
    \end{LTXexample}
  \end{example}
\end{frame}

\begin{frame}
  \frametitle{Mathematics in \LaTeX}

  \begin{itemize}
    \item \LaTeX{} provides a powerful and flexible environment for typesetting mathematical expressions.
    \item Whether you need inline math for brief equations within text or display math for standalone equations, \LaTeX{} has you covered.
    \item This slide explores the basics of including mathematical content in your \LaTeX{} documents.
  \end{itemize}
\end{frame}

\begin{frame}[fragile]{Inline Math Mode}
  Use single dollar signs (\$) for inline math mode. For example:

\begin{example}
\begin{LTXexample}
The Pythagorean theorem is $a^2 + b^2 = c^2$.
\end{LTXexample}
\end{example}
\end{frame}

\begin{frame}[fragile]{Display Math Mode}
  Use double dollar signs (\$\$) or \texttt{\textbackslash[} and \texttt{\textbackslash]} for display math mode. For example:

\begin{example}
\begin{LTXexample}
$$E = mc^2$$

or

\[\int_{a}^{b} f(x) \,dx = F(b) - F(a)\]
\end{LTXexample}
\end{example}
\end{frame}

\begin{frame}{Math Symbols and Operators}
  \LaTeX{} provides a wide range of symbols and operators for mathematical expressions.

  \begin{example}
    \begin{itemize}
      \item Greek letters: $\alpha, \beta, \gamma$
      \item Relations: $a < b, c \geq d, e \neq f$
      \item Operators: $\sum_{i=1}^{n} x_i, \int_{a}^{b} f(x) \,dx$
      \item Fractions: $\frac{a}{b}$
      \item Roots: $\sqrt{2}, \sqrt[n]{x}$
    \end{itemize}
  \end{example}
\end{frame}

\begin{frame}[fragile]{Equation Numbering}
  Use the \texttt{equation} environment for automatically numbered equations.

\begin{example}
\begin{LTXexample}
\begin{equation}
  x^2 + y^2 = r^2
\end{equation}
\end{LTXexample}
\end{example}

  To prevent numbering, use \texttt{\textbackslash begin\{equation*\}} and \texttt{\textbackslash end\{equation*\}}.
\end{frame}

\begin{frame}{Table of contents}
  In \LaTeX, the table of contents (TOC) is generated automatically using the \texttt{\textbackslash tableofcontents} command.

  \begin{itemize}
    \item Place \texttt{\textbackslash tableofcontents} at the point in your document where you want the TOC to appear.
    \item \LaTeX{} will automatically populate the TOC based on your document's structure.
  \end{itemize}
\end{frame}

\begin{frame}
  \frametitle{Including Images in \LaTeX}

  \begin{itemize}
    \item \LaTeX{} allows seamless inclusion of images to enhance visual content.
    \item The \texttt{graphicx} package provides essential tools for handling images.
    \item This slide explores how to effectively include and customize images in your \LaTeX{} documents or presentations.
  \end{itemize}
\end{frame}

\begin{frame}{The \texttt{graphicx} Package}
  The \texttt{graphicx} package provides the \texttt{\textbackslash includegraphics} command to include images in your \LaTeX{} document.

  \begin{example}
    \begin{itemize}
      \item[] \texttt{\textbackslash usepackage\{graphicx\}}
      \item[] \texttt{\textbackslash includegraphics[options]\{filename\}}
    \end{itemize}
  \end{example}

  Options can include width, height, scale, and more.
\end{frame}

\begin{frame}{Supported Image Formats}
  \LaTeX{} supports various image formats, including:

  \begin{itemize}
    \item PNG
    \item JPEG
    \item PDF
    \item EPS
    \item ... and more
  \end{itemize}
\end{frame}

\begin{frame}{Including an Image}
  To include an image, use the \texttt{\textbackslash includegraphics} command:

  \begin{example}
    \begin{itemize}
      \item[] \texttt{\textbackslash includegraphics[width=0.8\textbackslash textwidth]\{image.png\}}
    \end{itemize}
  \end{example}

  This example scales the image to 80 percent of the text width.
\end{frame}

\begin{frame}{Figure Environment}
  Wrap the \texttt{\textbackslash includegraphics} command in the \texttt{figure} environment for better image placement and to add captions.

  \begin{example}
    \begin{itemize}
      \item[] \texttt{\textbackslash begin\{figure\}}
      \item[] \texttt{\textbackslash centering}
      \item[] \texttt{\textbackslash includegraphics[width=0.8\textbackslash textwidth]\{image.png\}}
      \item[] \texttt{\textbackslash caption\{A descriptive caption.\}}
      \item[] \texttt{\textbackslash end\{figure\}}
    \end{itemize}
  \end{example}
\end{frame}

\begin{frame}[fragile]{Image Demo}

\begin{LTXexample}
\begin{figure}
  \centering
  \includegraphics[width=0.4\textwidth]{Tux.png}
  \caption{A lovely penguin!}
\end{figure}
\end{LTXexample}

\end{frame}

\begin{frame}[fragile]{Image positioning}
This figure environment supports positions as its extra options:
\lstinline!\begin{figure}[h]\end{figure}!
  
\begin{table}
  \caption{Different options}  \begin{tabular}{c|c}
    h & Try to place it here \\
    t & Top of the page\\
    b & Bottom of the page\\
    p & Put it on a separate page
  \end{tabular}
\end{table}

\end{frame}

\begin{frame}{Introduction to Tables}
  Tables are essential for organizing and presenting data in a structured manner. In \LaTeX{}, you can create tables to suit various needs in documents or presentations.
\end{frame}

\begin{frame}{Basic Table Structure}
  The basic structure of a table in \LaTeX{} involves the use of the \texttt{tabular} environment:

  \begin{example}
    \begin{itemize}
      \item[] \texttt{\textbackslash begin\{tabular\}\{clr\}}
      \item[] \texttt{Content \& More \& Text \textbackslash\textbackslash}
      \item[] \texttt{\textbackslash end\{tabular\}}
    \end{itemize}
  \end{example}
\end{frame}

\begin{frame}{Basic Table Structure}
  \begin{figure}
  \begin{tabular}{clr}
    Content & More & Text \\
  \end{tabular}
  \caption{Previous table}
  \end{figure}
\end{frame}


\begin{frame}{Table Columns}
  Specify the number and alignment of columns in the \texttt{tabular} environment:

  \begin{example}
    \begin{itemize}
      \item[] \texttt{\textbackslash begin\{tabular\}\{lcc\} \% Alignment for each column}
      \item[] \texttt{Left \& Center \& Center \textbackslash\textbackslash}
      \item[] \texttt{\textbackslash end\{tabular\}}
    \end{itemize}
  \end{example}
\end{frame}

\begin{frame}{Table Columns}
  \begin{figure}
  \begin{tabular}{lcc}
    Left & Center & Center \\
  \end{tabular}
  \caption{Previous table}
\end{figure}
\end{frame}

\begin{frame}{Adding Horizontal Lines}
  Use \texttt{\textbackslash hline} to add horizontal lines to separate rows in a table:

  \begin{example}
    \begin{itemize}
      \item[] \texttt{\textbackslash begin\{tabular\}\{clr\}}
      \item[] \texttt{\textbackslash hline}
      \item[] \texttt{Content \& More \& Text \textbackslash\textbackslash}
      \item[] \texttt{\textbackslash hline}
      \item[] \texttt{Content \& More \& Text \textbackslash\textbackslash}
      \item[] \texttt{\textbackslash hline}
      \item[] \texttt{\textbackslash end\{tabular\}}
    \end{itemize}
  \end{example}
\end{frame}

\begin{frame}{Adding Horizontal Lines}
  \begin{figure}
  \begin{tabular}{clr}
    \hline
    Content & More & Text \\
    \hline
    Content & More & Text \\
    \hline
  \end{tabular}
  \caption{Previous table}
\end{figure}
\end{frame}

\begin{frame}{Adding Vertical Lines}
  Use vertical bars to add vertical lines between columns:

  \begin{example}
    \begin{itemize}
      \item[] \texttt{\textbackslash begin\{tabular\}\{|c|l|r|\}}
      \item[] \texttt{Content \& More \& Text \textbackslash\textbackslash}
      \item[] \texttt{\textbackslash end\{tabular\}}
    \end{itemize}
  \end{example}
  \caption{Previous table}
\end{frame}

\begin{frame}[fragile]{Labeling in \LaTeX}
  The \texttt{\textbackslash label\{key\}} command is used to label elements in \LaTeX, such as sections, figures, and tables.
  This allows you to reference these elements later.

\begin{example}
\begin{lstlisting}
\section{Introduction}
\label{sec:intro}
Content related to the section of intro...
\end{lstlisting}
\end{example}

\end{frame}

\begin{frame}{Referencing in \LaTeX}
  To refer to labeled elements, use the \texttt{\textbackslash ref\{key\}} command.

  \begin{example}
    \begin{itemize}
      \item[] \texttt{As discussed in Section \textbackslash ref\{sec:intro\}, ...}
      \item[] \texttt{Figure \textbackslash ref\{fig:example\} illustrates ...}
    \end{itemize}
  \end{example}
\end{frame}

\begin{frame}{Cross-Referencing with \texttt{hyperref}}

  The \texttt{hyperref} package enhances cross-referencing by creating hyperlinks. Include it in the preamble:

  \begin{example}
    \begin{itemize}
      \item[] \texttt{\textbackslash usepackage\{hyperref\}}
    \end{itemize}
  \end{example}

  Now, when you use \texttt{\textbackslash ref}, the text becomes clickable in the PDF.
\end{frame}

\begin{frame}
  \frametitle{Headers and Footers in \LaTeX}

  \begin{itemize}
    \item Headers and footers provide a way to customize the top and bottom parts of your document.
    \item In \LaTeX, you can control these elements using packages and commands.
    \item We will explore the customization of headers and footers, including predefined styles and the use of the \texttt{fancyhdr} package.
  \end{itemize}
\end{frame}

\begin{frame}{Page Styles}
  \LaTeX{} provides several predefined page styles, including:

  \begin{itemize}
    \item \texttt{plain}
    \item \texttt{headings}
    \item \texttt{empty}
  \end{itemize}

  These styles define the layout of headers and footers on different pages. and they can be used by
  specifying the style in the preamble using \texttt{\textbackslash pagestyle}.
\end{frame}


\begin{frame}{Customizing Headers and Footers}
  The \texttt{fancyhdr} package is commonly used to customize headers and footers. It provides commands like:

  \begin{itemize}
    \item \texttt{\textbackslash pagestyle\{fancy\}}
    \item \texttt{\textbackslash fancyhead\{...\}}
    \item \texttt{\textbackslash fancyfoot\{...\}}
  \end{itemize}

  Customize content, positions, and formatting as needed.
\end{frame}

\begin{frame}{Example: Custom Header and Footer}
  Example of customizing headers and footers using \texttt{fancyhdr}:

  \begin{example}
    \begin{itemize}
      \item[] \texttt{\textbackslash usepackage\{fancyhdr\}}
      \item[] \texttt{\textbackslash pagestyle\{fancy\}}
      \item[] \texttt{\textbackslash fancyhead[Location]\{...\}}
      \item[] \texttt{\textbackslash fancyfoot[Location]\{...\}}
    \end{itemize}
  \end{example}

  The location value can be \texttt{L, C, R} or more!
\end{frame}

\end{document}
