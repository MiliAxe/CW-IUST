\documentclass{beamer}

\usepackage{showexpl}
\lstset
{
    language=[LaTeX]TeX,
    breaklines=true,
    basicstyle=\tt\scriptsize,
    keywordstyle=\color{blue},
    identifierstyle=\color{magenta},
    numberstyle=\tiny\color{gray},
    tabsize=2,
    numbers=left,                    
    numbersep=5pt,
}


\title{Computer Workshop}
\subtitle{An intro to the web}
\author{Dr. MalekiMajd}
\institute[IUST]{Iran University Of Science And Technology}
\date{6 Dey, 1402}

\usetheme{Madrid}

\begin{document}

\begin{frame}
    \maketitle
\end{frame}

\begin{frame}{What to expect}
\begin{enumerate}
    \item Have an introduction to the web
    \item Learn about the web's history
    \item Get to know GitHub pages
    \item Setup a simple personal website
\end{enumerate} 
\end{frame}

\begin{frame}{What's happening in the background}
    Ever wondered what happens when you type in a URL and hit enter?
    
    There are a lot of things happening in the background, but we'll focus on the most important ones.
\end{frame}

\begin{frame}{What's happening in the background}
    \begin{enumerate}
        \item \textbf{DNS Resolutition}: The URL is translated to an IP address.
        \item \textbf{Establishing a connection}: A TCP connection is established between the client and the server.
        \item \textbf{Sending the request}: The client sends a request to the server.
        \item \textbf{Processing the request}: The server processes the request and sends a response.
        \item \textbf{Sending the response}: The server sends the response to the client.
        \item \textbf{Rendering the response}: The client renders the response.
    \end{enumerate}
\end{frame}

\begin{frame}{Note on the procedure}
    The procedure is not always the same. For example, the client might cache the response and not send a request to the server.

    Or many other scenarios that we won't get into right now. They will be covered in your assignment.
\end{frame}

\begin{frame}{A little bit of history}
    \begin{itemize}
        \item The web was invented by Tim Berners-Lee in 1989.
        \item Tim Berners-Lee, a British scientist at CERN, proposed the concept of the web as a way to share information among researchers.
        \item He developed the first web browser, called WorldWideWeb, and the first web server.
        \item The web was initially designed as a simple system for linking documents using hypertext.
        \item It quickly gained popularity and became the foundation of the modern internet.
    \end{itemize}
\end{frame}

\begin{frame}{What is an IP?}
    An IP (Internet Protocol) address is a unique numerical label assigned to each device connected to a computer network that uses the Internet Protocol for communication.
    It serves two main purposes: identifying the host or network interface and providing the location of the device in the network.
\end{frame}

\begin{frame}{What is a URL?}
    A URL (Uniform Resource Locator) is a unique identifier used to locate a resource on the internet. It's a string of characters that contains information about the location of a resource on the internet.
\end{frame}

\begin{frame}{How are URLs translated to IPs?}
    The process of translating a URL to an IP address is called DNS resolution. DNS stands for Domain Name System. It's a system that translates domain names to IP addresses.
\end{frame}

\begin{frame}{Ports}
    Ports are endpoints for communication in a computer network, allowing different applications to send and receive data.
    They are identified by numbers and are used to route network traffic to the appropriate application or service.
    \begin{itemize}
        \item Ports facilitate communication between different applications in a computer network.
        \item Each port is identified by a number, ranging from 0 to 65535.
        \item Ports are used to route network traffic to the appropriate application or service running on a device.
        \item Commonly used ports include port 80 for HTTP, port 443 for HTTPS, and port 22 for SSH.
    \end{itemize}
\end{frame}

\begin{frame}{What are web servers?}
    \begin{itemize}
        \item A web server is a computer that stores web pages and responds to requests from clients.
        \item You may have heard of Apache, Nginx, IIS, etc. These are all web servers.
        \item You may purchase a web server and host your website on it. But there are other options.
    \end{itemize}
\end{frame}

\begin{frame}{GitHub Pages comes into play!}
    \begin{itemize}
        \item GitHub Pages is a service offered by GitHub that allows you to host your website on GitHub's servers.
        \item It's free!
        \item It's easy to use!
        \item It's fast!
    \end{itemize}
\end{frame}

\begin{frame}{What we will be focusing on?}
    One of our goals is to learn how to use GitHub Pages to host our \textbf{portofolio} website. 
\end{frame}

\begin{frame}{What is a portofolio website?}
    A portofolio website is a website that showcases your skills and projects. It's a great way to show off your skills to potential employers.
\end{frame}

\begin{frame}{Why bother with a portofolio website?}
    \begin{itemize}
        \item It's a great way to present yourself in a professional manner.
        \item It's a great way to show off your skills.
        \item It's a great way to show off your projects.
        \item It's a great way to get hired/interned.
    \end{itemize}
\end{frame}

\begin{frame}{Let's get started}
    As mentioned, we will be using github pages to host our website.

    \begin{itemize}
        \item First, you need to have a GitHub account. If you don't have one, create one.
        \item Second, you need to create a repository named \textbf{username.github.io} where \textbf{username} is your GitHub username.
        \item GitHub will automatically host your website at \textbf{username.github.io}.
    \end{itemize}
\end{frame}

\begin{frame}{Getting the repository content}
    You might ask where I can get the content of the repository. Well, you can either:
    \begin{itemize}
        \item Create the content yourself.
        \item Use a template.
    \end{itemize}
    We will be focusing on a famous template. But you can use any template you want.

    Also, in case we had the time, we will be focusing on how to create the content ourselves.
\end{frame}

\begin{frame}{Jekyll}
    Jekyll is a static site generator. It's a tool that allows you to create static websites from templates.

    It's the tool that GitHub uses to generate GitHub Pages.
\end{frame}

\begin{frame}{al-folio}
    \textbf{al-folio} is a simple, clean, and responsive Jekyll theme for academics:
    
    \begin{itemize}
        \item It provides a simple way to create a personal website which is suitable for academics.
        \item It's easy to use.
        \item It's easy to customize.
        \item It's easy to deploy.
    \end{itemize}
\end{frame}

\begin{frame}{Getting started with al-folio}
    The recommended approach for using al-folio is to first create your own site using the template with as few changes as possible, and only when it is up and running customize it however you like. 
\end{frame}

\begin{frame}{Steps to install}
    This part is taken from the \textbf{al-folio} repository:
    \begin{enumerate}
        \item Create a new repository using this template. For this, click on \textbf{Use this template - Create a new repository} above the file list.
        \item In this new repository, go to \textbf{Settings - Actions - General - Workflow permissions} and give Read and write permissions to GitHub Actions.
        \item Open file \texttt{\_config.yml}, set url to \url{https://<your-github-username>.github.io} and leave baseurl empty.
        \item Finally, in the repository page go to \textbf{Settings - Pages - Build and deployment}, make sure that \textbf{Source} is set to \textbf{Deploy from a branch} and set the branch to \textbf{gh-pages} (NOT to master).
        \end{enumerate}
\end{frame}

\begin{frame}{The deployment process}
    The deployment process is simple. You just need to push your changes to the repository and GitHub will automatically deploy your website.
\end{frame}

\begin{frame}{Customizing the website}
    You are better off cloning the repository and customizing it locally. You can use any text editor you want. This makes the process much easier and hassle free.
\end{frame}

\begin{frame}{The config}
    You may notice that there is a file named \texttt{\_config.yml}. This file contains the configuration of the website. You can change the configuration to your liking.

    You may add your social media links, change the title of the website, change the description, etc.
\end{frame}

\begin{frame}{The content}
    The content of the website is located in the \texttt{\_pages} directory. You can change the content of the pages to your liking.

    You may add your own pages, remove the existing pages, etc.
\end{frame}

\begin{frame}{The posts}
    The posts of the website are located in the \texttt{\_posts} directory. You can change the content of the posts to your liking.

    You may add your own posts, remove the existing posts, etc.
\end{frame}

\end{document}